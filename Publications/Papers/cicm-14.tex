\documentclass{llncs}
\usepackage{url}

\newcommand{\SD}{{\sc SymbolicData}}

\title{The {\SD} Project}
\author{Hans-Gert Gr\"abe \and Andreas Nareike}
\institute{Universit\"at Leipzig, Germany\\
\email{(graebe|nareike)@informatik.uni-leipzig.de}}
\begin{document}
\maketitle

\begin{abstract}% 86 words
  We report about a complete redesign of the tools and data of the {\SD}
  project according to Linked Data principles and RDF technologies that proved
  to be powerful within modern semantic web approaches. During that redesign
  the focus of the project changed from a merely data store towards the vision
  of a Computer Algebra Social Network that aims at technical support of the
  intercommunity communication between different CA subcommunities. The
  redesign was released as version 3 of SymbolicData in September 2013 and
  since then steadily evolved.
\end{abstract}

\section{Introduction}

The SymbolicData project grew up from the Special Session on Benchmarking at
the 1998 ISSAC conference to continue the efforts started by the PoSSo
\cite{PoSSo} an FRISCO \cite{FRISCO} projects. It aimed at building a reliable
and sustainably available reference of Polynomial Systems data, to extend and
update it, to collect meta information about the records, and also to develop
tools to manage the data and to set up and run tests and benchmark
computations on the data.  A first prototype was developed during 1999--2002
by Olaf Bachmann and Hans-Gert Gr\"abe, that collected data from
\emph{Polynomial Systems Solving} and \emph{Geometry Theorem Proving}.

Olaf Bachmann left the project for a new job at the end of 2000, and there was
almost no advance during 2002--2005.  In a second phase around 2006 the
project matured again. Data were supplied by the CoCoA group (F.~Cioffi), the
Singular group (M.~Dengel, M.~Brickenstein, S.~Steidel, M.~Wenk),
V.~Levandovskyy (non commutative polynomial systems, G-Algebras) and R.
Hemmecke (Test sets from Integer Programming). In 2005 the German Fachgruppe
Computeralgebra launched the Web site \url{http://www.symbolicdata.org}.
During the Special Semester on Groebner Bases in March 2006 we tried to join
forces with the GB-Bibliography project (B. Buchberger, A. Zapletal) and the
GB-Facilities project (V. Levandovskyy).

In 2009 we started to refactor the data along standard Semantic Web concepts
based on the Resource Description Framework (RDF).  In 2012 these efforts were
supported by a 12 months grant \emph{Benchmarking in Symbolic Computations and
  Web 3.0} for Andreas Nareike within the \emph{Saxonian E-Science Initiative}
\cite{E-Science-Sachsen}.  Within that scope we completed a redesign of the
data distinguishing more consequently between data (\emph{resources} in the
RDF terminology) and meta data (\emph{knowledge bases} in the RDF terminology)
and refactoring the meta data along Linked Data principles.  The new {\SD}
data and tools were released as version~3 in September 2013.

\section{The {\SD} Infrastructure}

Our resources (examples for testing, profiling and benchmarking software and
algorithms from different areas of symbolic computation) are publicly
available in XML markup, meta data in RDF notation both from a public git
repo, hosted at \texttt{github.org}, and from an OntoWiki \cite{OntoWiki}
based data store at \url{http://symbolicdata.org/Data}.  Moreover, we offer a
SPARQL endpoint \cite{sdsparql} to explore the data by standard Linked Data
methods.

The website operates on a standardized installation using an Apache web server
to deliver the data, the Virtuoso RDF data store \cite{Virtuoso} as data
backend, a SPARQL endpoint and (optionally) OntoWiki to explore, display and
edit the data.  This installation can easily be rolled out at a local site
(tested under Linux Debian and Ubuntu 12.04 LTS; a more detailed description
can be found in the {\SD} wiki \cite{sdwiki}) to support local testing,
profiling and benchmarking.

The distribution offers also tools for integration with a local compute
environment as, e.g., provided by Sagemath \cite{Sagemath} -- the Python based
\emph{SDEval package} \cite{sdeval} by Albert Heinle offers a JUnit like
framework to set up, run, log, monitor and interrupt testing and benchmarking
computations, and the \emph{SDSage package} \cite{sdsage} by Andreas Nareike
provides a showcase for {\SD} integration with the Sagemath compute
environment.

We follow a development process along the Integration-Manager-Workflow Model.
This makes it easy to join forces with the {\SD} team: Fork the repo to your
github account, start development and send a pull request to the Integration
Manager if you think you produced something worth to be integrated into the
upstream master branch.  Even if your contribution is not pulled to the
upstream, people can use it, since they can pull it from your to their github
repo. This allows even for agile common small feature development -- a widely
practised way to advance projects hosted at \texttt{github.com}. You are
encouraged early to start a discussion about your plans and regularly report
your progress on the {\SD} mailing list.

\section{{\SD} Resources and Maintenance}

Preparing {\SD} version 3 we decided to strengthen the {\SD} part that is
\emph{not} involved with Polynomial Systems Solving.  These efforts led to a
more consequent distinction between data (owned and maintained by different CA
subcommunities) and meta data. Such a distinction is well supported by RDF
design principles -- the Resource Description Framework is about
\emph{description} of \emph{resources}, represented by (globally unique)
\emph{resource identifiers} (URIs). We follow the Linked Data best practise to
provide URIs in such a way, that they are accessible by the HTTP internet
protocol and a valuable part of structured information about that resource is
delivered upon HTTP request to that URI.

Currently the {\SD} data collection contains resources from the areas of
Polynomial Systems Solving (390 records, 633 configurations), Free Algebras
(83 records), G-Algebras (8 records), GeoProofSchemes (297 records) and Test
Sets from Integer Programming (28 records). These resources are stored in a
flat XSchema based XML syntax using well established intracommunity syntaxes
for the internal data.

Such a concept is not restricted to centrally managed resources, but can
easily be extended to other data stores on the web that are operated by
different CA subcommunities and offer a minimum of Linked Data facilities.
There are draft versions of resource descriptions about Fano Polytopes (8630
records) and Birkhoff Polytopes (5399 records) hosted by Andreas Paffenholz
and about Transitive Groups (3605 records) from the Database for Number Fields
of J\"urgen Kl\"uners and Gunter Malle that point to external resources.  This
part of our project requires further solicitation.

It was one of the great visions of the {\SD} Project to collect not only
benchmark and testing data but also valuable background information about the
records in the database as, e.g., information about papers, people, history,
systems etc.\ concerned with the examples in our collection.  Within the
redesign we developed a general concept of the \texttt{sd:Annotation} RDF
class to store background information in a unified way. 

We use that concept in particular to relate bibliographical entries of type
\texttt{sd:Reference} to different data records.  The management of
bibliographical references was completely redesigned with {\SD} version 3
exploiting RDF and the established Dublin Core ontology \cite{dcterms} to
represent bibliographical information in a way that is queryable by standard
means and tools. On the other hand, we strongly reduced the part of
information about bibliographical references kept inside {\SD} since there are
comprehensive bibliographical stores available on the web that provide all
required information.  

\section{Towards a Computer Algebra Social Network}

From the five stars to be assigned to a Linked Data project according to Tim
Berners-Lee's classification \cite{5stars} {\SD} earned four stars so far (for
offering data in interoperable RDF format on the web and providing a SPARQL
querable triple store).  For the fifth star one has to build up stable
semantic relations to foreign knowledge bases and thus become part of the
Linked Open Data Cloud \cite{lod}.

Much of such interrelation, e.g., a list of interoperability references for
people, software and bibliographical data with the Zentralblatt, is on the
way.  Moreover, we joined forces with the efforts of the board of the German
Fachgruppe to store and provide information about people and groups working on
CA topics at their new Wordpress driven web site \cite{cafg}.  We developed a
first prototype to store this information in RDF format, to extract it by
means of SPARQL queries and to view it on the web site using the Wordpress
shortcode mechanism via a special Wordpress plugin.  We apply the same
technique to maintain information about upcoming conferences at this site.

The vision of a Computer Algebra Social Network goes far beyond that: Set up
and run within the CA community a semantic aware Facebook like Social Network
and contribute to it about all topics around Computer Algebra using tools that
express your contributions in an RDF based syntax that the community agreed
upon. This sounds quite visionary, but it is in no way utopic. We operate a
first prototypical node of a tool that realizes the challenging concept of a
\emph{Distributed Semantic Social Network} \cite{dssn}, see our wiki for more
information.  

Due to page restrictions we cannot explain this and the new RDF based
techniques in more detail and refer to the extended version \cite{car-54} of
this paper.

\raggedright
\begin{thebibliography}{1}

\bibitem{5stars} Berners-Lee, T.: 5 stars for Open Data.
  \url{http://5stardata.info/} [2014-03-05]

\bibitem{dcterms} DCMI Metadata Terms.
  \url{http://dublincore.org/documents/dcmi-terms/} [2014-02-27]

\bibitem{E-Science-Sachsen} Das eScience-Forschungsnetzwerk Sachsen.
  \newblock \url{http://www.escience-sachsen.de} [2014-02-19]

\bibitem{cafg} Website of the German Fachgruppe Computeralgebra.  
  \url{http://www.fachgruppe-computeralgebra.de/} [2014-03-06]

\bibitem{FRISCO} FRISCO -- A Framework for Integrated Symbolic/Numeric
  Computation. \url{http://www.nag.co.uk/projects/FRISCO.html} [2014-02-19]

\bibitem{car-54} Gr\"abe, H.-G., Nareike, A.: The {\SD} Project
  -- from Data Store to Computer Algebra Social Network. 
  Computeralgebra-Rundbrief {\bf 54} (2014, to appear).

\bibitem{sdeval} Heinle, A.: The SDEval framework.  
  \url{http://symbolicdata.org/wiki/SDEval} [2014-02-28]

\bibitem{lod} Linked Data.  \url{http://en.wikipedia.org/wiki/Linked_data}
  [2014-03-05]

\bibitem{sdsage} Nareike, A.: The SDSage package.  
  \url{http://symbolicdata.org/wiki/PolynomialSystems.Sage} [2014-02-28]

\bibitem{OntoWiki} OntoWiki: A tool providing support for agile, distributed
  knowledge engineering scenarios. 
  \url{http://aksw.org/Projects/OntoWiki.html} [2014-02-19]

\bibitem{PoSSo} The PoSSo Project. \url{http://posso.dm.unipi.it/}
  [2014-02-19]

\bibitem{Sagemath} Sage -- a free open-source mathematics software system.
  \url{http://www.sagemath.org} [2014-02-19]

\bibitem{sdwiki} The {\SD} project wiki.  \url{http://symbolicdata.org/wiki}

\bibitem{sdsparql} The {\SD} SPARQL endpoint.
  \url{http://symbolicdata.org:8890/sparql}

\bibitem{dssn} Tramp, S.\ et al.: DSSN: towards a global Distributed Semantic
  Social Network.  \url{http://aksw.org/Projects/DSSN.html} [2014-03-06]

\bibitem{Virtuoso} Virtuoso Open-Source Edition. \newblock
  \url{http://virtuoso.openlinksw.com/} [2014-02-19]

\end{thebibliography}
\end{document}
