\documentclass{slides}
\usepackage{url}
\newcommand{\SD}{{\sc Symbolic\-Data}}
\newcommand{\titel}[1]{\begin{center}\large\bf #1\end{center}}

\begin{document}
\sloppy
\begin{center}

{\Large \bf The \SD\ Project} 

{\small\it Towards an Electronic Repository of Tools and Data for
Benchmarks of Computer Algebra Software}
\vskip3cm

Olaf Bachmann (Kaiserslautern, Germany)\\
Hans-Gert Gr\"abe (Leipzig, Germany)

on behalf of the \SD\ project team
\vskip3cm

\url{http://www.SymbolicData.org}
\end{center}
\pagebreak

\titel{Motivation}

For further qualification of benchmark efforts in Computer Algebra it
would be of great benefit to {\em unify} the different benchmark
approaches and to {\em systematically collect} the existing special
and general benchmark data such that they are {\em electronically
available} in a more or less uniform way. This would yield an
electronic repository of certified inputs and results that could be
addressed and extended during further development. 

The \SD\ project is set out to approve concepts, develop tools, and
collect data as a first step towards such a far reaching undertaking.

\pagebreak

The \SD\  project has the following three goals:
\begin{enumerate}
\item To systematically collect existing symbolic computation
  benchmark data and to produce tools with which this data collection
  can conveniently be extended and maintained.
\item To design and implement concepts which facilitate trusted
  benchmark computations on the collected data.
\item To provide tools that allow to access, select and extract data
using different technologies (Perl, SQL, WWW, etc) and to convert data
into commonly used formats, e.g., HTML, ASCII, LaTeX, etc. or prepare
them for input to different Computer Algebra software.
\end{enumerate}
\pagebreak

\titel{The Design of the Data Base}

An {\bf object-relational data base concept} does not only allow to
systematically collect and store data, but also offers concepts to
interrelate different data, e.g., problem descriptions, computational
results, background information, citations, and to design modular,
object-orien\-ted tools for data access and manipulations.

We decided (at least at the moment) not to use one of the various data
base programs but to {\bf keep the primary sources in an XML-like
ASCII format} stored/retrieved through {\bf Perl tools}. We call such
files {\bf sd-files} and their associated records {\bf sd-records} and
use them as the basic units to store all information. 

Similar records share a common structure and, as in the ER model, are
grouped into {\bf tables}.

{\bf Meta tables} are used to specify and define, through special
sd-records, the syntax and semantics of the tags of data tables and
also some {\bf class attributes} that specify properties of entire
data tables.

There exists a well defined {\bf type concept} for tag values.

{\bf Interrelations} between tables are specified by means of tags of
the type {\tt Ref} as hashes of key/comment pairs where `key' is the
name of a record, or even a regular expression matching several
records, in the foreign table and `comment' is any text.

Interrelations are used to attach to a record, for example,
bibliography entries (from the {\tt BIB} table), problem descriptions
(from the {\tt PROBLEMS} table) etc.

\pagebreak

\titel{The Design of the Perl Tools}

Requirements:
%\vspace{-35pt}

\begin{itemize}\setlength{\itemsep}{-15pt}
\item operations for very different tasks 
\item as simple and as flexible as possible
\item extendability and reusability
\end{itemize}

We provide
\begin{enumerate}
\item a highly modularized {\bf programming environment} 
for independent and rapid development of new components and
specialized applications

and 

\item a well-documented, flexible, and intuitive {\bf standard
interface program} for most of the implemented operations.
\end{enumerate}
\pagebreak

{\bf Basic modules}\\ implement primitive operations, like I/O
and tag/value access of sd-records.

{\bf Action modules}\\ implement the generic part of actions like
validate, insert, compute, transform, \ldots
 
{\bf Table modules}\\ implement table specific parts of actions.

{\bf The {\tt symbolicdata} program}\\ is the standard interface. It
realizes command-line parsing, initialization of global variables and
required modules, and execution of the actions inherited from the
command line.

Adequate and up-to-date {\bf documentation} is kept closely to the
source code and can be extracted by appropriate Perl tools.
\pagebreak

\titel{Collecting and Maintaining Data}

To collect data from a certain application field one first has to {\bf
specify the structure of the records to be collected}. This requires
to create one or several data tables via their meta tables.

A {\bf meta table} consists of a set of tag descriptions, i.e.,
sd-files, that can be created with any text editor.  Each such meta
sd-file contains the description of the attributes of a tag of the
table to be defined.

Predefined tags, PERSON tag.

{\bf Specific Perl functions} might have to be implemented and
specified in the meta sd-records which realize semantical operations
like validation, generation, and comparison of tag values.

After meta specification, records of this table may be {\bf inserted}
into the data base. 

Each record has to be supplied as sd-file that either can be created
by a text editor from a template or converted with appropriate Perl
tools, possibly using the \SD\ programming environment, from other
formats.

Records are {\bf validated} and {\bf checked for uniqueness} during
insertion to guarantee consistency of the database.  

Evaluating semantical aspects requires to cooperate with software
capable of symbolic manipulations.
\pagebreak

\titel{Running Benchmark Computations}
\SD's {\tt Compute} environment is set out to realize the following
three goals:
\begin{enumerate}
\item To facilitate automated and trusted benchmark computations.
\item To serve as a test-bed for developers, that is, as a tool with
which developers of Computer Algebra software can easily evaluate new
algorithms and implementation techniques.
\item To provide a repository of computational results which can be
used for further development, like computing invariants of the
original example, correctness verifications and timing comparisons of
other computations, etc.
\end{enumerate}

Benchmark computations reveal dependencies on the following parameters

The {\bf Example}\\ which is to be computed, i.e., an sd-record from a
benchmark collection which provides the object of the computation.

{\bf COMP}\\ The actual computation to be performed, i.e., an
sd-record of type {\tt COMP} which describes the computation and
serves as an interface to (Perl) routines, which examine an example
for suitability for this computation, and, where applicable, check
the syntactical and semantical correctness of the result of the
computation.

{\bf CASCONFIG}\\ A configuration of a Computer Algebra software which
realizes the computation, i.e., an sd-record of type {\tt CASCONFIG}
which identifies the software, its version, and its implemented
benchmark capabilities, \\[12pt]
and serves as an interface to (Perl) routines which generate the input
file and shell command to run the computation, check the output of the
computation for run-time errors, and, if necessary, perform
(syntactic) transformations on the result such that it is suitable for
further processing independent of the examined Computer Algebra
software.

{\bf MACHINE}\\ A description of the computer used for the
computation.

{\bf Dynamic parameters}\\ This includes specifications of: intervals
for the run-time of a computation; which error, resp. verification,
checks should be performed on the result; what to do with the output
of the computation.

\pagebreak

\titel{The Current State of the Project}

The \SD\ project evolved as a permanent interplay between its two
facets: collecting data and extending/improving concepts, design, and
tools.

As of today, the \SD\ contributors collected more than 1100
sd-records, wrote 40 Perl modules with more than 15\,000 lines of
code, and implemented 22 actions for the standard interface program
{\tt symbolicdata}.  


Short alphabetical overview of tables which currently exist:

{\bf Table BIB}: Bibliography entries. 

{\bf Table CAS}: General descriptions of Computer Algebra software.

{\bf Table CASCONFIG}: Configurations of Computer Algebra software to
execute benchmarks.

{\bf Table COMP}: Descriptions of computations.

{\bf Table COMPREPORT}: Reports of executed benchmark computations.

{\bf Table COMPRESULTS}: Output of executed benchmark computations.

{\bf Table GEO}: A benchmark collection of problems arising in
mechanized geometry theorem proving.

{\bf Table INTPS}: A benchmark collection of polynomial systems with
integer coefficients.

{\bf Table MACHINE}: Computers on which benchmark computations are
performed.

{\bf Table PERSON}: Developers/contributors who are involved with \SD.

{\bf Table PROBLEMS}: More detailed background information and
comments about different problems.


We started first benchmark computations on Groebner bases, using
various coefficient domains and monomial orderings.  These benchmarks
have been (and are) run on the more than 500 {\tt INTPS} records using
10 versions of different Computer Algebra systems. 

The web site \url{www.SymbolicData.org} is the central site of the
\SD\ project, containing its WWW-pages, and its CVS and FTP
repositories.  

\pagebreak

\titel{The Project Team}

\begin{quote}\small 
Olaf Bachmann (Kaiserslautern)\\ 
Michael Dengel (Saarbr\"ucken)\\
Jean-Charles Faugere (Paris)\\
Hans-Gert Gr\"abe (Leipzig)\\
Hans Sch\"onemann (Kaiserslautern)\\

We used also material compiled by 

Malte Witte (Leipzig)\\
Ralf Hemmecke (Leipzig, now Linz)
\end{quote}




\titel{Thanks}

\begin{quote}\small 
To the UMS Medicis for providing the hardware and
software to set up \url{www.SymbolicData.org} and for letting us use
their excellent computing facilities.

To the Fachgruppe Computeralgebra of the Deutsche Mathematiker
Vereinigung for sponsoring the \url{www.SymbolicData.org} domain.
\end{quote}


\end{document}
