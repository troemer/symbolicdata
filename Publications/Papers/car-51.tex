\documentclass{article}
\usepackage{a4wide,german,url}
\usepackage[utf8]{inputenc}

\newcommand{\SD}{{\sc Symbolic\-Data}}
\parindent0pt
\parskip4pt

\begin{document}

\section*{\centering Neues vom \SD-Projekt}

\begin{center} 
Hans-Gert Gr"abe (Leipzig)\footnote{erscheint im ``Computeralgebra-Rundbrief''
  51, Oktober 2012}
\end{center}

\subsection*{Zur Geschichte des Projekts}

Die Idee zum \SD-Projekt entstand 1998 am Rande der ISSAC in Rostock, als
zunehmend sichtbar wurde, dass die Datensammlungen, welche im Rahmen der
Projekte PoSSo\footnote{\url{http://posso.dm.unipi.it}} und
FRISCO\footnote{\url{http://www.nag.co.uk/projects/frisco.html}} entstanden
waren, ohne aktives Zutun der Community nicht dauerhaft digital verfügbar sein
würden.

In den folgenden Jahren wurde dieses Projekt vor allem von Olaf Bachmann und
dem Autor dieses Beitrags mit weiterer Unterstützung aus dem Umfeld der
Singular-Gruppe vorangetrieben. Im CAR 26 (M"arz 2000) und CAR 28 (M"arz 2001)
wurde das \SD-Projekt im damaligen Zuschnitt ausf"uhrlich vorgestellt.  Der
Schwerpunkt der Datensammlung lag im Bereich polynomialer Gleichungssysteme
mit ganzzahligen Koeffizienten, die zu jener Zeit insbesondere im Zusammenhang
mit Varianten des Gröbner-Algorithmus intensiv studiert wurden, wobei die
verschiedenen Benchmark-Beispiele aus der Literatur und den genannten
EU-Projekten systematisiert und digital aufbereitet sowie entsprechende
Werkzeuge zur Verwaltung und zum Aufsetzen von Benchmarks entwickelt wurden.
 
Nach dem Weggang von O.~Bachmann aus Kaiserslautern Ende 2000 wurde mit dem
geometrischen Theorembeweisen ein zweites größeres Sammelgebiet erschlossen,
das über die Koordinatenmethode eng mit polynomialen Gleichungssystemen
verbunden ist. Einfache Beispiele lassen sich auf geometrische Sätze \emph{vom
  konstruktiven Typ} zurückführen, deren Beweis auf eine
{\glqq}einfache{\grqq} Normalformberechnung rationaler Ausdrücke zurückgeführt
werden kann. Obwohl solche Normalformberechnungen im Prinzip verstanden sind,
laufen sie sich aber oft praktisch nicht durch, während alternative
Koordinatisierungen \emph{vom Gleichungstyp} und geschickte Ansätze des Lösens
der entsprechenden polynomialen Gleichungssysteme zum Ziel führen. M.~Witte,
damals Student in Leipzig, untersuchte und digitalisierte insbesondere eine
größere Anzahl von Beispielen aus dem Buch von S.-C.~Chou\footnote{S.-C.~Chou.
  Mechanical Geometry Theorem Proving. Reidel, Dortrecht, 1988.}.

Weiteres Beispielmaterial wurde von der CoCoA-Gruppe (F.~Cioffi), der
Singular-Gruppe (M.~Dengel, M.~Brickenstein, S.~Steidel, M.~Wenk),
J.-C.~Faugère (weitere Polynomsysteme), V.~Levandovskyy (nichtkommutative
Polynomsysteme, G-Algebren) sowie von Raymond Hemmecke (Testmengen aus der
ganzzahligen Programmierung) beigesteuert.  Die Projektdaten wurden über ein
CVS-Repository verwaltet.  Die Domäne \texttt{symbolicdata.org} wude und wird
durch die CA-Fachgruppe finanziert und seit 2005 von der GI-Geschäftsstelle
gehostet.

Leider standen ab 2003 keine personellen Ressourcen zur systematischen
weiteren Arbeit am \SD-Projekt zur Verfügung, so dass das Datenmaterial
seither allein auf einer {\glqq}as is{\grqq} Basis zur Verfügung stand und
auch nur gelegentliche Erweiterung erfahren hat. Insbesondere stellte sich
heraus, dass die ambitionierten Vorstellungen, eine \emph{einheitliche
  Werkbank für Benchmarking im symbolischen Rechnen} mit entsprechenden
Werkzeugen und Konzepten zu entwickeln, mit den verfügbaren Ressourcen nicht
zu stemmen war und -- mit Blick auf die Vielfalt der Rechner\-architekturen,
Problemstellungen und verwendeten Programmiersprachen -- wohl auch am realen
Bedarf vorbei ging. Statt dessen konzentriert sich dieser Teil des Projekts
nunmehr darauf, den Code von Best Practise Beispielen zu sammeln und zur
Nachnutzung zur Verfügung zu stellen.

Ein weiterer Versuch, die Arbeiten am \SD-Projekt mit vereinten Anstrengungen
zu intensivieren und mit angrenzenden Bemühungen in dieser Richtung, etwa der
Datenbank \emph{Gröbner Bases Implementations. Functionality Check and
  Comparison}\footnote{\url{http://www.risc.jku.at/Groebner-Bases-Implementations}}
oder dem \emph{Gröbner Bases Bibliography
  Project}\footnote{\url{http://www.risc.jku.at/Groebner-Bases-Bibliography}}
zu koordinieren, wurde im März 2006 während des \emph{Special Semester on
  Groebner Bases} am RISC Linz gestartet.  Jedoch fand auch dieses Projekt im
Nachgang keine nachhaltige Resonanz in der Community. 

\subsection*{Entwicklungen nach 2003 -- {\SD} meets RDF}

In den letzten 10 Jahren nahm das (semantische) Web der Daten eine besonders
stürmische Entwicklung, was die konzeptionellen Entwicklungen zur Datenhaltung
aus den Anfangszeiten des \SD-Projekts entwertet hat, über die seinerzeit in
den entsprechenden Aufsätzen im CAR berichtet wurde.  Mit dem \emph{Semantic
  Stack} haben sich inzwischen Standards durchgesetzt und stehen Werkzeuge zur
Verfügung, mit denen sich Verwaltungsaufgaben auf verschiedenen Ebenen nach
einheitlichen Prinzipien organisieren lassen. Dies sind 
\begin{itemize}\raggedright
\item XML und XSchema als Standards zur Darstellung strukturierter Daten auf
  Ressourcen-Ebene sowie 
\item RDF und OWL als Standards zur Beschreibung semantischer Aspekte der
  Daten.
\end{itemize}
Während ersteres prinzipiell geeignet ist, dezentral betreute Datenbestände in
einem \SD-Netz zusammenzuführen und durch geeignete URI's zu adressieren (die
Frage der \emph{nachhaltigen} Verfügbarkeit derartiger dezentral betreuter
Datenbestände sei hier ausgeklammert), ist die Aggregierung der Beschreibungen
semantischer Aspekte unbedingt erforderlich, um sinnvolle Suchprozesse nach
Informationen auf diesen vernetzten Beschreibungen zu ermöglichen.  Während
wir in der Anfangszeit davon ausgingen, dass dazu ein wesentlicher Teil der
Informationen auch an einem Ort zusammengetragen sein muss -- etwa als
Datenbestand in unserem CVS-Repository -- zeigen modernere Entwicklungen der
Linked Open Data Cloud, in welchem Umfang auch diese Informationen dezentral
verwaltet werden können, wenn entsprechende Protokolle des Datenaustauschs
vereinbart werden. Genauer geht es nicht um die Protokolle selbst, die als
\emph{Metasprache} mit dem RDF-Framework und dessen Erweiterungen weitgehend
standardisiert sind, sondern um die gemeinsam zu vereinbarende Ontologie als
\emph{domänenspezifische Sprache}, mit der RDF-Tripel der Wissensbasis auf
einheitliche Weise {\glqq}verstanden{\grqq} werden.

Die wenigen verfügbaren Ressourcen der letzten Jahre wurden darauf verwendet,
einen Migrationsprozess der Datenbasis hin zu diesen Standards in die Wege zu
leiten, also 
\begin{itemize}\raggedright
\item die Benchmark-Daten aus den verschiedenen Bereichen auf je
  standardisierte Weise in XSchema-beschriebene XML-Ressourcen zu
  transformieren, 
\item die im ursprünglichen \SD-Konzept eng mit den Ressourcen gekoppelten
  Beschreibungsdaten zu separieren und über RDF-Standards (insbesondere einen
  SPARQL-Endpunkt\footnote{Provisorisch unter \url{http://hgg.ontowiki.net}
    verfügbar.}) für Suchanfragen verfügbar zu machen sowie
\item eine \SD-Ontologie als domänenspezifische vom Computer lesbare Sprache
  für diese Beschreibungsdaten zu entwickeln.
\end{itemize}
Die Erfahrungen aus vergleichbaren Projekten der
AKSW-Gruppe\footnote{\url{http://wiki.aksw.org/About}} an unserem Lehrstuhl
legten nahe, dass dieser Migrationsprozess am besten als agiler Prozess
anzulegen ist, also das Prinzip des {\glqq}a little bit thinking, a little bit
coding{\grqq} aus der Anfangszeit des \SD-Projekts auch hierbei anwendbar ist,
um mit kleinen Schritten die sehr begrenzten Ressourcen einzusetzen.
Insbesondere lehren diese Beispiele, dass es -- im deutlichen Gegensatz zu den
Erfahrungen der ER-Modellierung -- nicht sinnvoll ist, bereits am Anfang zu
viel Mühe auf die genaue Fassung der \SD-Ontologie zu verwenden. Hierbei wird
der Vorteil von RDF gegenüber herkömmlichen Datenbankansätzen deutlich, denn
Prädikate lassen sich auf einfache Weise in der Wissensbasis selbst hinzufügen
oder redefinieren, was bei Datenbanken ein aufwändiges Redesign der
Tabellenstruktur erfordert. Die Konsolidierung der Ontologie selbst kann so
auf einen späteren Zeitpunkt verschoben werden, wenn eine ausreichende Zahl
von {\glqq}Sätzen{\grqq} der zu definierenden Sprache bereits vorliegen.

Im Repository sind größere Teile der Daten (insbesondere ein wesentlicher Teil
der Polynomsysteme) inzwischen in XSchema-beschriebene XML-Ressourcen
transformiert. Etwas schwieriger gestaltet sich der Transformationsprozess bei
den Geometrie-Beispielen, da dort zusätzlich die Ähnlichkeiten zu den
Beschreibungssprachen verschiedener DGS, also neue semantische Aspekte,
berücksichtigt werden sollen.

Anfang 2011 wurde die Projektverwaltung\footnote{Siehe
  \url{https://bitbucket.org/hgg/symbolicdata}} auf Mercurial umgestellt, da
so Entwicklungszweige deutlich flexibler angelegt und wieder mit dem
Hauptstamm vereinigt werden können.  In einem solchen Zweig\footnote{Siehe
  \url{https://bitbucket.org/levandov/symbolicdata_freealgebra}} wurden
weitere Beispiele zu freien Algebren zusammengetragen.

\subsection*{Das Benchmarking-Projekt im Rahmen der\\ Sächsischen
  E-Science-Initiative} 

Im Rahmen der Sächsischen
E-Science-Initiative\footnote{\url{http://www.escience-sachsen.de}} wurde ein
Projekt {\glqq}Benchmarking im symbolischen Rechnen{\grqq} bewilligt, über das
ab September 2012 für 12 Monate eine halbe Mitarbeiterstelle finanziert und
mit Andreas Nareike besetzt ist.  Damit besteht die Möglichkeit, die
bisherigen Migrationen zu konsolidieren und dem \SD-Projekt, wenigstens für
eine gewisse Zeit, neuen Schwung zu verleihen.

Wir möchten dabei weitere interessierte Partner und Datenbestände einbeziehen
und insbesondere die Beschreibungstechniken auf RDF-Ebene weiter
qualifizieren.  Zur Intensivierung der Kommunikation haben wir dazu neben der
schon länger existierenden Mailingliste \texttt{symbolicdata} bei Google
Groups ein Mediawiki\footnote{\url{http://symbolicdata.uni-leipzig.de}} an der
Uni Leipzig aufgesetzt, in das die bisherigen Inhalte der Webseiten sowie
weitere Informationen migriert werden sollen.

Während eines
\begin{center}\bf
  Arbeitstreffens und Workshop am 13.--15. Dezember 2012 in Leipzig
\end{center}
sollen die bis dahin erreichten Ergebnisse vorgestellt und das weitere
Vorgehen abgestimmt werden. Dazu laden wir interessierte Mitstreiter ein.  Der
Workshop ist als {\glqq}Zero Budget Workshop{\grqq} geplant und wird
voraussichtlich an der HTWK Leipzig stattfinden.  Mehr dazu im \SD-Wiki.
\end{document}
