\documentclass[a4paper,11pt]{article}
\usepackage{CAR}

\begin{document}
\setcounter{footnote}{0}
\setcounter{figure}{0}

%%%%%%%%%%%%%%%%%%%%%%%%%%%%%%%%%%%%%%%%%%%%%%%%%%%%%%%%%%%%%%%%%%%%%%%%%%%%%%%%

\Abschnitt
% Name der Rubrik
{Rubrik}
% Name der Rubrik (Inhaltsverzeichnis)
{Rubrik}
% Label fuer die Rubrik
{rubrik}

\vspace{3mm}

%%%%%%%%%%%%%%%%%%%%%%%%%%%%%%%%%%%%%%%%%%%%%%%%%%%%%%%%%%%%%%%%%%%%%%%%%%%%%%%%

\Aufsatz
% Titel des Aufsatzes
{The {\SD} Project}
% Titel des Aufsatzes (Inhaltsverzeichnis)
{The {\SD} Project}
% Autor
{Hans-Gert Gr\"abe}
% Abkuerzung des Namens als Label
{H.-G. Gr\"abe}
% Autor mit Adresse
{H.-G. Gr\"abe\\(Universit"at Leipzig)}
% Dateiname eines Bildes
{nn}
% E-Mail
{graebe@informatik.uni-leipzig.de}

\vspace{3mm}

\begin{multicols}{2}
\noindent

%%%%%%%%%%%%%%%%%%%%%%%%%%%%%%%%%%%%%%%%%%%%%%%%%%%%%%%%%%%%%%%%%%%%%%%%%%%%%%%%

% Ueberschriften fuer Abschnitte im Aufsatz:
\Ueberschrift{Introduction}{intro}

In a more and more networked and interlinked world both the prospects and
importance of a well desgined and powerful digital research
\emph{infrastructure} dramatically increase. Well acknowledged efforts in that
direction are met for years within the organizational structures of the
scholarly process -- dissemination of new papers, refereeing processes,
scientific communication within communities. MathSciNet, ArXiv.org, well
established bibliographical services as, e.g., bibsonomy.org or \emph{The DBLP
  Computer Science Bibliography}, witness for these efforts.  With the
\emph{GND Project} (Gemeinsame Norm-Datei) the network of German Scientific
Libraries made strong efforts to unite and build up an integrated digital
information system that enhances the classical catalogue system of meta
information about scholarly work and makes it ready for the digital age.

Smaller academic communities, as, e.g., the Computer Algebra community, are
challenged in the same way to reorganize also their intracommunity
communication networks and infrastructure.  Its hard to allocate resources for
such a reorganization process since infrastructural efforts -- implementation
of new algorithms into computer programs, management of research databases,
preparation of a new release of a well established Computer Algebra System
etc. -- are rarely acknowledged by the reputational processes of the community
and hence are left to the casual engagement of volunteers if not supported by
the leading edge scientists of the community.  Such a misrecognition of
important efforts is addressed for years in particular in the Computer Algebra
Communities.

\Ueberschrift{The {\SD} Project at Large}{large}

{\SD} is such a project. It grew out from the special session on Benchmarking
at the 1998 ISSAC conference, where the community was faced with a typical
situation: Within the EU-supported PoSSo and FRISCO projects volunteers
compiled a large database of Polynomial Systems with the goal to make it
publicly available for testing and benchmarking of algorithms.  If the project
funding finished people switched to other tasks and it became more and more
problematic to access the data. Moreover, more or less well cloned copies of
the data went across the globe and after a while it was even hard to decide,
what, e.g., \emph{Katsura-5} means -- is it about the example from the well
known series with 5 variables $y_1,\dots,y_5$ or with 6 variables
$x_0,\dots,x_5$?

The {\SD} Project started in 1999 to build a reference for such a database of
Polynomial Systems, to extend and update it, to collect meta information about
the records, and also to develop tools to manage the data and to set up and
run testing and benchmarking computations on the data. The main design
decisions and implementations of the first prototype were realized by Olaf
Bachmann and Hans-Gert Gr\"abe in 1999 and 2000. We collected data from
\emph{Polynomial System Solving} and \emph{Geometry Theorem Proving}, set up a
CVS repository, and started test computations, both at UMS Medicis, with the
main focus on Polynomial System Solving. The prototype was presented at the
Meeting of the Fachgruppe Computeralgebra, Kaiserslautern, February 2000.

The project resources drastically shrinked when Olaf Bachmann left the project
for a new job at the end of 2000.  We presented the project within talks at
RWCA-02, ADG-02 and also in the CA-Rundbrief published by the Fachgruppe
Computeralgebra, but there was almost no advance of the project during
2002--2005.  In a second phase around 2006 the project matured again. Data
were supplied by the CoCoA group (F.~Cioffi), the Singular group (M.~Dengel,
M.~Brickenstein, S.~Steidel, M.~Wenk), V.~Levandovskyy (non commutative
polynomial systems, G-Algebras) and Raymond Hemmecke (Test sets from Integer
Programming). In 2005 the Web site \url{http://www.symbolicdata.org} was set
up by the German \emph{Fachgruppe Computeralgebra}. During the Special
Semester on Groebner Bases in March 2006 we tried to join forces with the
GB-Bibliography project (Bruno Buchberger, Alexander Zapletal) and the
GB-Facilities project (Viktor Levandovskyy). Unfortunately, all that turned
out to be another flash in the pan.

A third phase started in 2009 if we joined forces with the Agile Knowledge
Engineering and Semantic Web (AKSW) Group at Leipzig University to strongly
refactor the data along standard Semantic Web concepts based on the Resource
Description Framework (RDF).  In 2012 these efforts were supported by a 12
months grant \emph{Benchmarking in Symbolic Computations and Web 3.0} for
Andreas Nareike within the \emph{Saxonian E-Science Initiative}.

We completed a redesign of the data distinguishing more consequently between
data (\emph{resources} in the RDF terminology) and meta data (\emph{knowledge
  bases} in the RDF terminology) and refactoring the meta data along the
Linked Data principles.  The new {\SD} data and tools were released as
version~3 in September 2013.  

Resources (testing and benchmarking examples from different Computer Algebra
areas) are publicly available in XML markup, meta data in RDF notation both
from a public git repo, hosted at github.org, and from an OntoWiki based data
store at symbolicdata.org/Data.  Moreover, we offer a SPARQL endpoint at
symbolicdata.org to explore the data by standard Linked Data methods.

The website operates on a standard installation using an Apache web server to
deliver the data, the Virtuoso RDF data store as data backend and SPARQL
endpoint and (optionally) OntoWiki to explore, display and edit the data.
This standard installation can easily be rolled out at a local site (tested
under Linux Debian and Ubuntu 12.04 LTS; a more detailed description can be
found in the {\SD} wiki) to support local testing and benchmarking. 

The distribution offers also tools for such a local integration -- the Python
based \emph{SDEval package} by Albert Heinle (formerly Aachen, now at Uni
Waterloo) that offers a JUnit like framework to set up, run, log, monitor and
interrupt testing and benchmarking computations, and the \emph{sdsage package}
by Andreas Nareike, a {\SD} integration with the Sagemath Project. 

\Ueberschrift{The {\SD} Design}{design}

to be added.

%%%%%%%%%%%%%%%%%%%%%%%%%%%%%%%%%%%%%%%%%%%%%%%%%%%%%%%%%%%%%%%%%%%%%%%%%%%%%%%%

% Literaturverzeichnis
\begin{thebibliography}{1}
\itemsep=0cm plus 0pt minus 0pt

% Makro fuer einen Eintrag im Literaturverzeichnis
\bibitem
% Label
{B1}
% Autoren
Max~H.\ Musterman und Hans~M.\ N.N.
% Titel
\newblock Beispieltitel.
% Zeitschrift
\newblock {\em Computeralgebra-Rundbrief}, 54:0--1, March 2014.

\end{thebibliography}

%%%%%%%%%%%%%%%%%%%%%%%%%%%%%%%%%%%%%%%%%%%%%%%%%%%%%%%%%%%%%%%%%%%%%%%%%%%%%%%%

\end{multicols}
\end{document}


\documentclass{article}
\usepackage{a4wide,german,url}
\usepackage[utf8]{inputenc}

\newcommand{\SD}{{\sc Symbolic\-Data}}
\parindent0pt
\parskip4pt

\begin{document}

\section*{\centering Beitrag zum Teil „Berichte von Konferenzen“}

\subsection*{Workshop on SymbolicData Design}

Leipzig, 27. -- 28. August 2013

\url{http://symbolicdata.org/wiki/Events.2013-08}

The workshop was designed as final milestone of the E-Science Benchmarking
Project promoted for 12 month within the \emph{E-Science Saxony Framework}.
Unfortunately, the event was completely ignored by the Computer Algebra
Communities, so that we had no opportunity to present the results of the
project to a larger audience.  Instead we had intense discussions with people
from the \emph{swmath} project (\url{http://www.swmath.org}, a project of the
\emph{Zentralblatt Mathematik} towards an information service for mathematical
software) about trends in Semantic Web Technologies that are suitable to
support future common efforts towards a semantic aware IT infrastructure for
Computer Algebra.

In a first talk \emph{Hans-Gert Gräbe} presented the state of the SymbolicData
project.  Note that at the end of September 2013 version 3 of SymbolicData was
released, thus finishing a major redesign of SymbolicData, that marks a
milestone across the implementation of semantic techniques within Computer
Algebra.  We strongly use RDF and Linked Data principles in the organisation
of the data. These principles are also reflected in the presentation of the
data at \texttt{symbolicdata.org}. All resources are delivered via
\texttt{http/rdf+xml} and a Sparql endpoint allows for navigation in the
metadata. This can be installed also on a localhost and thus can be integrated
into a local benchmarking or profiling infrastructure (best using python as
scripting language and a web server at localhost). A more detailed description
of the new release is available from the SymbolicData web pages and will be
given also in the next issue of the \emph{Computeralgebra Rundbrief}.

\emph{Andreas Nareike} presented in a second talk his prototypical integration
of the Polynomial Systems subproject with sagemath and SymbolicData as a sage
package \emph{sdsage} that smoothly integrates both the global SD network
infrastructure and a local installation into the sagemath process. One can
load data and metadata transparently into sage objects and process them as
mathematical objects in the usual way within sage. 

\emph{Ulf Schöneberg} gave a talk about effort at the ZBMath to discover and
understand mathematical formulas in Zentralblatt mathematical reviews, mixing
classical colocation approaches with semantic enriched opportunities of latex
mark-up. This research is part of larger efforts within, e.g., the OpenMath
activities.

We discussed in great detail the potential interplay between
\begin{itemize}
\item the efforts at ZBMath to organize access to data in well established RDF
  based formats,
\item the SymbolicData intercommunity efforts and experience with Linked Data
  standards, Sparql endpoints, Virtuoso and Ontowiki based local
  installations,
\item ongoing efforts of the DNB and other libraries (SLUB Dresden, UB
  Leipzig) to reshape their catalogue data towards Linked Data standards and
  get them interoperating within the GND project,
\item perspectives to join forces with these library projects to strengthen
  the IT infrastructure for Computer Algebra Communities.
\end{itemize}
\begin{flushright}
    Hans-Gert Gr"abe (Leipzig)\footnote{to be published in
      „Computeralgebra-Rundbrief“ 53, Oktober 2013}
  \end{flushright}
\end{document}
