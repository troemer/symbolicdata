\documentclass{beamer}
\usepackage{german}
\usepackage[english]{babel}
% Designelemente
\usetheme{Boadilla}
\beamertemplatenavigationsymbolsempty

\newcommand{\ueberschrift}[1]{\begin{center}\bf\large #1\end{center}}
\newenvironment{code}{\footnotesize\tt \begin{tabbing}
\hskip12pt\=\hskip12pt\=\hskip12pt\=\hskip12pt\=\hskip5cm\=\hskip5cm\=\kill}
{\end{tabbing}}

\title[The SymbolicData Project]{The SymbolicData Project\\ Towards a
  Computer Algebra Social Network}

\subtitle{Talk given at the International Conference on\\ Infrastructures and
  Cooperation in e-Science and e-Humanities}

\author[Gr\"abe, Johanning]{Hans-Gert Gr\"abe, Simon Johanning}

\institute[Uni Leipzig]{Leipzig University, Germany\\
\texttt{http://bis.informatik.uni-leipzig.de/HansGertGraebe}} 
\titlegraphic{\includegraphics[width=.3\textwidth]{Logos/escience.png}}

\date{Leipzig, 2014-06-05}
\begin{document}
\begin{frame}
\maketitle
\end{frame}
\begin{frame}{Aim and Scope}{}
Vision:
\begin{itemize}
\item Develop concepts and tools for profiling, testing and benchmarking
  Computer Algebra Software (CAS) from different areas of Computer Algebra
\item Collect and interlink relevant data and activities from different
  Computer Algebra subcommunities
\end{itemize}
{SymbolicData is an}
\begin{itemize}
\item inter-community project that has its roots in the activities of
  different Computer Algebra Communities and
\item aims at interlinking these activities using modern Semantic Web
  concepts. 
\end{itemize}
{Tools and data are designed to be used both}
\begin{itemize}
\item on a local site for special testing and profiling purposes
\item and to manage a central repository at http://www.symbolicdata.org.
\end{itemize}
\end{frame}

\begin{frame}{What does SymbolicData offer?}{}
Data:
\begin{itemize}
\item Polynomial Systems Solving
\item Geometry Theorem Proving
\item Fano Polytopes (A. Paffenholz)
\item Free Algebras
\item G-Algebras
\item Test Sets from Integer Programming
\end{itemize}
{Draft:}
\begin{itemize}
\item Birkhoff Polytopes (A. Paffenholz)
\item Transitive Groups (J. Kl\"uners, G. Malle)
\end{itemize}
\end{frame}

\begin{frame}{What does SymbolicData offer?}{}
{Tools:}
\begin{itemize}
\item SDEval Package (Albert Heinle)
\begin{itemize}
\item Aim: Set up, run, log, monitor standardized Computations on SD data
  series in a reliable way 
\item Technology: Python standalone on top of the OS
\end{itemize}
\item SDSage Package (Andreas Nareike)
\begin{itemize}
\item Aim: Call the new Polynomial Systems format from Sagemath 
\item Technology: Sagemath Python Package
\end{itemize}
\end{itemize}
I will not touch that topic in my talk.
\end{frame}

\begin{frame}{What does SymbolicData offer?}{}
{Infrastructure:}
\begin{itemize}
\item Github repositories (following the Integration Master Pattern)
\item A project wiki at http://symbolicdata.org
\item A mailing list
\item Web access to the XML resources
\item A centrally operated OntoWiki based RDF data store of meta informations
  based on the Virtuoso RDF store
\item Organized along Linked Data Principles
\item Regular dumps of RDF data in Turtle format
\item A SPARQL endpoint to query the data
\item Advise for easy local installation of tools and data based on Virtuoso
  and a local Apache Web server (OntoWiki optional)
\end{itemize}
\end{frame}

\begin{frame}{Some History}{}
ISSAC 1998: Special session on Benchmarking 

1999-2002: Phase 1 -- Olaf Bachmann, Hans-Gert Gr\"abe
\begin{itemize}
\item Focus: Polynomial Systems, tools and concepts
\item Technology: XML-like special markup, elaborated Perl tools
\end{itemize}
2005-2007: Phase 2 -- around the Groebner Special Year in Linz
\begin{itemize}
\item Focus: Geometry Theorem Proving, first interlinking projects with the GB
  bibliography and the GB facilities projects
\item Technology: Switch to true XML concepts
\end{itemize}
2012-2014: Phase 3 -- E-Science Saxonia supported project (Andreas Nareike,
Hans-Gert Gr\"abe, Simon Johanning)
\begin{itemize}
\item Focus: Switch to Linked Data and Semantic Web concepts, XML resources,
  RDF meta data, data reorganization
\item Release of version 3 in Sept. 2013
\end{itemize}
\end{frame}

\begin{frame}{Linked Data Principles}{}
\begin{itemize}
\item \emph{Resources:} URI, HTTP Get access
\begin{itemize}
\item URI = Unique Resource Identifier
\item Access to worldwide distributed data in a unified way
\end{itemize}
\item \emph{Resource Descriptions:} Deliver a valuable piece of information in
  structured RDF format, that can be combined with other pieces of information
  from other sources into new RDF sentences.
\begin{itemize}
\item RDF = Resource Description Framework
\end{itemize}
\item Run \emph{RDF Triple Stores} as part of a worldwide distributed data
  storage infrastructure
\begin{itemize}
\item Triple: »subject predicate object.« as the basic RDF information unit.
\end{itemize}
\item (Federated) Query Language SPARQL
\item Run \emph{SPARQL Endpoints} on RDF triple stores
\end{itemize}
\end{frame}

\begin{frame}{RDF Basics}{}
Main idea: Store pieces of information as triples.
\begin{itemize}
\item Subject and predicate have to be URIs, object can be an URI or a literal
  (i.e., plain or typed text)
\item Different ASCII storage formats (RDF-XML, JSON, Turtle) and tools to
  parse and transform these formats
\item We use the Redland RDF Libraries http://librdf.org/
\item Representation as (directed) RDF graph: Subjects and objects as nodes
  (literals as annotated blank nodes), predicates as (labelled) edges.
\end{itemize}
Allows for navigation within the data: SPARQL Query Language

RDF allows to describe Resources, Concepts (i.e., meta information about
Resources), Ontologies (i.e., meta information about Concepts) etc.\ in a
uniform way.
\end{frame}

\begin{frame}{SymbolicData Data Structure}{}
Resources:
\begin{itemize}
\item SD provides own resources in an XML based format
\begin{itemize}
\item Polynomial Systems, Geometry Theorem Proving, \ldots
\end{itemize}
\item Draft: SD addresses other resources at different stores
\begin{itemize}
\item Polytopes, Transitive Groups
\end{itemize}
\item Maintenance of resources requires special semantic knowledge, semantic
  aware tools and semantically educated people
\end{itemize}
Resource Descriptions: 
\begin{itemize}
\item Precomputed \emph{fingerprints} of the different resources in RDF format
  to navigate and search within the data.\\ It requires \emph{semantic}
  knowledge both to compute fingerprints and to use them in an appropriate
  way.
\end{itemize}
\end{frame}

\begin{frame}{SymbolicData Data Structure}{}
Resource Descriptions (cont.)
\begin{itemize}
\item \emph{Background information:} Use RDF to manage additional data, try to
interlink that data with other sources along the Linked Data Principles.
\begin{itemize}
\item Annotations -- a notational system to associate background information
  to different examples and series of examples
\end{itemize}
\item Bibliography -- bibliographical references system (to be aligned with
  ZBMath)
\item People -- different people and groups (to be aligned with ZBMath)
\item Systems -- list of CA software (aligned with swmath) 
\end{itemize}
\end{frame}

\begin{frame}{Towards a CA Social Network}{}
Valuable background information is information the people care about. 
\begin{itemize}
\item Find out the places where such information is spread today.\\ Usually it
  is \emph{streamed}, not \emph{stored}.
\item Try to semantically annotate that information. 
\item Build views (web sites) that harvest information.
\end{itemize}
Vision:
\begin{itemize}
\item People -- enlarge the database, link it to the ZBMath people database
\begin{itemize}
\item Used to display people from the CAFG Board within the Wordpress based
  CAFG site
\end{itemize}
\item Groups and Projects -- collect standard information about CA working
  groups and their projects
\begin{itemize}
\item Used to display such information within the Wordpress based CAFG site
\end{itemize}
\end{itemize}
\end{frame}

\begin{frame}{Towards a CA Social Network}{}
Vision (continued):
\begin{itemize}
\item Conferences -- do not only send conference announcements around mailing
  lists, but store it in a commonly agreed format within a CA Social Network
\begin{itemize}
\item A very first prototype is used to display such information within the
  Wordpress based CAFG site
\end{itemize}
\item The stakeholders understand, that this is a techno-social, and even more
  a social than a technical process that is best discussed on the Symbolicdata
  Mailing list.
\item The CASN germ at http://symbolicdata.org/wiki/CASN matures thanks to
  common efforts.
\end{itemize}
\end{frame}

\begin{frame}{Links}{}\small
\begin{itemize}
\item \texttt{http://symbolicdata.org} -- the SD Wiki
\item \texttt{http://symbolicdata.org/XMLResources} -- the SD XML Resources
\item \texttt{http://symbolicdata.org/RDFData} -- the SD RDF Data Turtle Files
\item \texttt{http://symbolicdata.org/Data} -- the SD OntoWiki view on RDF
  data
\item \texttt{https://github.com/symbolicdata} -- the SD Repository at github
\end{itemize}
\end{frame}

\end{document}


